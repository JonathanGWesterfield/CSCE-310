\documentclass [letter,12pt] {article}
\usepackage{fullpage}
\usepackage{graphicx}
\usepackage[shortlabels]{enumitem}
\begin{document}
\title{
    \huge{\textbf{Homework 9}}
}
\author{
    Jonathan Westerfield \\
    224005649
}
\date{\today}
\maketitle

\begin{quote}
\centering
\textit {
    An Aggie Does Not Lie Cheat Or Steal Nor Tolerate Those Who Do. \\
}
\vspace {1.4in}
\hrulefill
\end{quote}
\newpage

\section{SimpleDB Where Clause}
    In SimpleDB, a valid SQL statement is SELECT sname FROM  student; which does not
    have a WHERE clause. However, SelectScan requires a Predicate. Explain how
    SimpleDB successfully implements this situation.
    \\
    \\
    SimpleDB executes the select statement successfully in the parser by setting 
    the required predicate to Null or an empty String when there is no WHERE clause
    This then returns all of the rows that were given since there is no longer
    a limiting condition for the query.
    
    
\section{Exercise 15.3.2 p.722}
    Suppose B(R) = B(S) = 10,000, and M = 1000. Calculate the disk I/O cost
    of a nested-loop join.
    \\
    \\
    \[\frac{B(S)(M-1 + B(R))}{M-1} = \frac{1000((1000-1) + 10000)}{1000-1} = 110100\] 
    \\
    \begin{center}
        There would be 110100 Disk I/O
    \end{center}
    
    
\section{MethodName()}
    Place a System.out.println("methodName()"); statement at the beginning of every
    method in the class Parser and every "eat" method of class Lexar - changing the
    "methodName()" printed to the name of the method called. Run the following test
    program:
    \begin{verbatim}
package simpledb.parse; public class TestParser 
{
    public static void main(String[] args) 
    {
        Parser parser = new Parser("delete from xyz where a = b and c = 0");
        DeleteData data = parser.delete();
        System.out.println("table: "+data.tableName().toString());
        System.out.println("Pred: "+data.pred().toString());
    } 
}
    \end{verbatim}
    
    Using the output of the test program, create a parse tree or the given SQL command.
    \\
    \\
    \begin{figure}[htp]
        \centering
        \includegraphics[width=\textwidth]{Parser_Tree.png}
        \caption{Parse Tree of the Above Code}
        \label{fig:updateplanner}
    \end{figure}
    
    
\section{Transaction Method nextTxNumber Synchronized}
    Explanin why the Transaction class method nextTxNumber is synchronized. What 
    could happen if it were not? Is it a problem for the other methods in that class
    to not be synchronized? Explain your answer.
    \\
    \\
    The nextTxNumber() method is synchronized in order to prevent race
    conditions. The Transaction class is supposed to be used for concurrent
    operations as seen by the use of the ConcurrencyMgr type. Keeping
    nextTxNumber synchronized is used in order to make sure two or more 
    running threads to not attempt to increment the txNumber at the same
    time. If they were to do so, it would be impossible to determine the 
    behavior of the program since any data type that is being used 
    concurrently without protection can cause very strange and unpredictable
    results. There are multiple transactions being created at the same due
    to the possible multitude of transactions a user can run. This keeps 
    the program flow predictable by controlling how the transaction number 
    is updated.
    
    
\section{SimpleDB Update Planner}
    The SimpleDB update planner doesn't verify that the assignment of a new value to
    the specified field in a modify statement is type-correct. For example, in the 
    ChangeMajor.java program, if we had "UPDATE student SET MajorId=`30', a problem 
    happens during the execution of the command because MajorId is an integer in the
    student table and not a string. Demonstrate the problem and turn-in a screen 
    shot of the stack trace when the error happens. After looking at the error,
    check the server. You’ll find that it is still operational. So only the client 
    code was terminated. Philosophically, as a systems developer, would you fix the
    code in the TableScan.setVal() method, throw an SQLException or do nothing (let 
    the user program terminate with the ClassCastException? Explain your answer.
    \\
    \\
    \begin{figure}[htp]
        \centering
        \includegraphics[width=\textwidth]{Update_Planner.png}
        \caption{Exception Thrown By the Update Planner}
        \label{fig:updateplanner}
    \end{figure}
    
    I would personally throw a SQLException for a couple of reasons. First, throwing
    a SQLException would closer mimic production database systems such a MySQL and 
    MariaDB. When trying to insert the wrong data type into a field, it will not 
    allow you do so and will throw an exception. Secondly, there are just too many
    variables at play and too many assumptions that we would need to make about 
    what the user was attempting to do. There is a possibility the user intentionally 
    did this but it is also very likely the user made a mistake and we don't want to 
    allow that mistake to propagate. It would be very bad if we assumed the user meant 
    add the number 30, change setVal() to convert to a string and upload that to the 
    SimpleDB database, and had the user \textit{actually} attempt to put a string 
    literal into the MajorID field. Allowing the client to rollback instead of aborting 
    would be a better way to go about it. The user can fix the mistake and
    run the query again.
    
    
\end{document}