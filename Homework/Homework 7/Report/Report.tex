\documentclass [letter,12pt] {article}
\usepackage{fullpage}
\usepackage{graphicx}
\usepackage[shortlabels]{enumitem}
\begin{document}
\title{
    \huge{Homework 7}
}
\author{
    Jonathan Westerfield \\
    224005649
}
\date{\today}
\maketitle

\begin{quote}
\centering
\textit {
    An Aggie Does Not Lie Cheat Or Steal Nor Tolerate Those Who Do. \\
}
\vspace {1.4in}
\hrulefill
\end{quote}
\newpage

\section{FileMgr.java}
    Attached is a version of FileMgr.java where I added instrumentation 
    to track the number of block reads and writes, accessor methods for 
    these values and methods to reset them. Use your ninja Java skills 
    to modify the methods \textit{commit()} and \textit{rollback()} in 
    the class \textit{Transaction} 
    (in simpledb.remote) to print these statistics—note: you will need to 
    get the FileMgr object using the static method SimpleDB.fileMgr() in
    simpledb.server. Reset the counts when you print them. Start the 
    SimpleDB server and run the client program StudentMajor.java. Turn-in
    a copy of the output for the server console and explain the counts 
    printed and the counts output for the client program. Also, turn-in a 
    hardcopy of your modifications to Transaction.
    \\
    \\

\section{SimpleDB Block Size}
    Assume the SimpleDB block size of 400 given in class Page and that records
    cannot span blocks. Show the calculation for the maximum number of records
    that will fit in a SimpleDB record page and the amount of wasted 
    (fragmented) space in the page for each of the following record sizes
    given in bytes: 20, 50 and 100 (don’t forget to include the 4-byte 
    empty/in-use flag for each record). Also, explain how a SimpleDB record 
    file can have a block with no records in it.
    \\
    \\

\section{Transaction T}
    Consider the following scenario: Transaction T deletes a record from a
    block, setting a record’s flag to EMPTY. Transaction U inserts a 
    record into the block and the slot where the previously deleted 
    record is picked to hold the new record, setting its flag to INUSE. 
    Transaction U also writes new values into the fields of the new record. 
    Transaction T rolls back. Since the old values of the deleted record 
    were not saved in the log, the record cannot be restored to its original 
    value—ugh! a) Explain why this scenario cannot happen in SimpleDB. 
    b) If the transactions T and U attempt to perform the above steps, what 
    actually happens in SimpleDB?
    \\
    \\

\section{LogMgr.iterator}
    The code for LogMgr.iterator calls flush(). Explain why this call is 
    necessary, or not. Here is the code in question.

    \begin{verbatim}
        public synchronized Iterator<BasicLogRecord> iterator() {
            flush();
            return new LogIterator(currentblk);
        }
        private void flush() {
            mypage.write(currentblk);
        }
    \end{verbatim}
    IS THE CALL TO flush() NECESSARY? Explain your answer.
    \\
    \\

\section{SimpleDB Log Manager}
    The SimpleDB log manager implements the LSN of a log record as its 
    block number in the log file.

    \begin{enumerate}[(a)]
        \item Explain why this implementation of LSN does not uniquely
            identify the log records.
        \item Explain why this implementation may cause the method 
            LogMgr.flush to unnecessarily write the log manager’s page 
            to disk.
        \item Modify the code (and the implementation of an LSN) so that
            the page is written to disk only if absolutely necessary. 
            Turn-in a hardcopy of your code changes.
        \item How do you know if the code works? See if you can design 
            an experiment using the code you wrote for problem 1 above
            to prove that your modified code improves operation of the 
            log manager. Turn in your experiment and the output results 
            with an explanation.
    \end{enumerate}


\section{Textbook Exercise 17.1.2, p. 851, Part b.}

\section{Project Problem}
    The goal of this problem is to modify the Games collection by adding
    the corresponding player\_name to each game document. A game document
    contains the player\_id of the corresponding Profile document, but we
    want to add the player\_name from the Profile document to the game
    document. You can solve this problem two ways—via shell commands, or 
    a custom Java program. You decide. As proof of the success of your 
    solution turn in the output of a shell query on the games collection 
    using the player\_name (not the player\_id) that returns a count of the 
    documents in the Games collection for each of the players, “Tom Brady ”, 
    “Terry Bradshaw” and “Jim Brown”. Submit any java code to ecampus as 
    part of your ZIP file. You do not have to turn in a hard-copy of your 
    java code. If you solve it via shell commands, turn in a hardcopy of 
    the console output from running the commands.
    \\
    \\
    \begin{figure}[htp]
    \centering
    \includegraphics[width=\textwidth]{Tom Brady Count.png}
    \caption{Tom Brady Count}
    \label{fig:bradycount}
    \end{figure}
    
    \begin{figure}[htp]
    \centering
    \includegraphics[width=\textwidth]{Terry Bradshaw Count.png}
    \caption{Terry Bradshaw Count}
    \label{fig:bradshawcount}
    \end{figure}
    
    \begin{figure}[htp]
    \centering
    \includegraphics[width=\textwidth]{Jim Brown Count.png}
    \caption{Jim Brown Count}
    \label{fig:jimcount}
    \end{figure}
\end{document}