\documentclass [a4paper,12pt] {article}
\usepackage{fullpage}
\usepackage{graphicx}
\begin{document}
\title{
    \huge{Homework \& Project Work 6}
}
\author{
    Jonathan Westerfield \\
    224005649
}
\date{\today}
\maketitle

\begin{quote} 
\centering 
\textit {
    An Aggie Does Not Lie Cheat Or Steal Nor Tolerate Those Who Do. \\
}
\vspace {1.4in}
\hrulefill
\end{quote}
\newpage

\section {Exercise 13.2.5, p568}
    To modify a block on disk, we must read it into main memory, perform the 
    modification, and write it back. Assume that the modification in main memory 
    takes less time than it does for the disk to rotate, and that the disk 
    controller postpones other requests for disk access until the block is ready
    to be written back to the disk. For the Megatron 747 disk, what is the time
    to modify a block?

    \vspace{.2in}

    The formula for modification time is: seek time + average rotational latency 
    + transfer time. In this case, we have 2 rotations and are using the average 
    values for a write to the disk. The time is 
    \(6.46 + 4.17 + (2 * 8.33) +.13 = 27.42 ms\)
\section{Exercise 13.3.5, p575}
    If we read \textit{k} randomly chosen blocks form one cylinder, on the
    average how far around the cylinder must we go before we pass all of the 
    blocks?

    \vspace{.2in}

    1024 blocks are exactly one cylinder of the Megatron 747 and they can be 
    accessed in one average seek. All the blocks on a cylinder can be read in 
    16 rotations of the disk. The number of \textit{k} blocks would need to be
    divided by 1024 and then multiplied by 16.
    \[rotations = (k / 1024) * 16\]
\section{Exercise 13.4.1, p587}
    Compute the parity bit for the following bit sequences:
    \subsection {Part A: 00111011}
        001110111
    \subsection {Part B: 00000000}
        000000000
    \subsection {Part C: 10101101}
        101011011

\section{Exercise 13.4.7, p589}
    Using the same RAID level  4 scheme as in Exercise 13.4.6, suppose that data 
    disk 1 has failed. Recover the block of that disk under the following 
    circumstances:
    \subsection {Part A}
        The contents of disks 2 through 4 are 01010110, 11000000, and 00111011,
        while the redundant disk holds 11111011.

        \vspace{.2in}

        Disk 1: 01010110
    \subsection {Part B}
        The contents of disks 2 through 4 are 11110000, 11111000, and 00111111,
        while the redundant disk holds 00000001.
    
        \vspace{.2in}

        Disk 1: 00110110
\section{Exercise 13.5.1, p593}
    Suppose a record has the following fields in this order: A character string 
    of length 15 (15 bytes), an integer (2 bytes), a SQL date (10 bytes), and 
    a SQL time (10 bytes) (no decimal point). How many bytes does the record take if:
    \subsection {Part A}
        Fields can start at any byte.

        \vspace{.2in}

        Assuming that the SQL date is a CHAR array of length 10 for the format:
        HH:MI:SS (supported range is from  '-838:59:59' to '838:59:59'), then
        the record length is as follows: \(15 + 2 + 10 + 10 = 37 \) bytes.
    \subsection {Part C}
        Fields must start at a byte that is a multiple of 8.

        \vspace{.2in}

        Using the same information above, the character string must be of length
        16, the integer must be 8 bytes, the SQL date is 16 bytes and the SQL time
        is 16 bytes in order to be a multiple of 8. The record length is as follows:
        \(16 + 8 + 16 + 16 = 56\) bytes.

\section{Laptop Specifications}
    Concerning your laptop specifications. What is its processor and clock speed? 
    What is the size and speed of main memory? What is the size and speed of its 
    non-volatile storage device?

    \begin{itemize}
        \item[$-$]Processor: Intel Core i7
        \item[$-$]Processor Clock Speed: 2.2Ghz
        \item[$-$]Size of Main Memory: 16 GB
        \item[$-$]Speed of Main Memory: 1600 MHz DDR3
        \item[$-$]Size of Non-volatile Storage: 149 GB
        \item[$-$]Speed of Non-volatile Storage:
        \begin{itemize}
            \item[$-$]Read Speed: 710.4 MB/s
            \item[$-$]Write Speed: 281.9 MB/s
        \end{itemize}
    \end{itemize}

\section{Mongo Import}
    Use Mongoimport to make a collection of "plays" from the data in NFL Play by Play
    2009-2018(v5).csv. Turn in a screen shot of your import. Note: I renamed the file
    to plays.csv and used the mongo program:
    \begin{verbatim}
        mongoimport -d testdb -c plays --type csv --headerline -v --file 
        fileLocation
    \end{verbatim}

    \begin{figure}
        \includegraphics[width=\linewidth]{NFLPlaysImport.png}
        \caption{Importing the NFL Plays CSV File.}
        \label{fig:import1}
    \end{figure} 

    Figure \ref{fig:import1} below shows a screenshot of the NFL Plays CSV file import
    using the terminal command below:

    \begin{verbatim}
        mongoimport -d playsDB -c plays --type csv --headerline -v --file 
        /Users/JonathanWesterfield/Documents/CSCE\ 310/Homework/Homework\ 4
        /Java/Datasets/NFL2009-2018v5.csv
    \end{verbatim}

\section{Game Scoring Reports}
    \subsection{Game 2009091000}
        Figure \ref{fig:game2009091000} below shows the game report for game 2009091000.

    \subsection{Game 2017112000}
        Figure \ref{fig:game2017112000} below shows the game report for game 2017112000.

    \subsection{Game 2016101601}
        Figure \ref{fig:game2016101601} below shows the game report for game 2016101601.
    
    \subsection{Game 2018092303}
        Figure \ref{fig:game2018092303} below shows the game report for game 2018092303.

    \begin{figure}
        \includegraphics[width=\linewidth]{Game_Report_2009091000.png}
        \caption{Game Report for Game 2009091000}
        \label{fig:game2009091000}
    \end{figure} 
3
    \begin{figure}
        \includegraphics[width=\linewidth]{Game_Report_2017112000.png}
        \caption{Game Report for Game 2017112000}
        \label{fig:game2017112000}
    \end{figure} 

    \begin{figure}
        \includegraphics[width=\linewidth]{Game_Report_2016101601.png}
        \caption{Game Report for Game 2016101601}
        \label{fig:game2016101601}
    \end{figure} 

    \begin{figure}
        \includegraphics[width=\linewidth]{Game_Report_2018092303.png}
        \caption{Game Report for Game 2018092303}
        \label{fig:game2018092303}
    \end{figure} 


\end{document}